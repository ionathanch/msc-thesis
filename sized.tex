\chapter{Sized Dependent Types} \label{ch:sized-dep-types}

\newcommand{\FigSyntax}[1]{
  \begin{figure}[h]
  \centering
  \begin{align*}
  i, j, k, m, n &\Coloneqq \meta{\textrm{naturals}} &
  \Gamma &\Coloneqq \mt \mid \Gamma, \annot{x}{\tau} \mid \Gamma, \define{x}{\tau}{e} &
  r, s &\Coloneqq \alpha \mid \sss{s} \mid \circ \\
  f, g, x, y, z &\Coloneqq \meta{\textrm{term variables}} &
  \Delta &\Coloneqq \mt \mid \Delta, \annot{x}{\tau} &
  U &\Coloneqq \Prop \mid \Type{i} \\
  \alpha, \beta, \gamma &\Coloneqq \meta{\textrm{size variables}} &
  \Phi &\Coloneqq \mt \mid \Phi, \alpha \mid \Phi, \bound{\alpha}{s} \\
  e, P, \tau, \sigma &\Coloneqq \mathrlap{x \mid U \mid \funtype{x}{\tau}{\tau} \mid \fun{x}{\tau}{e} \mid \app{e}{e} \mid \letin{x}{\tau}{e}{e}} \\
  &\mid \mathrlap{\Funtype{\alpha}{\tau} \mid \Funtype<{\alpha}{s}{\tau} \mid \Fun{\alpha}{e} \mid \Fun<{\alpha}{s}{e} \mid \App{e}{s}}
  %&\mid \Pairtype{\alpha}{\tau} \mid \Pair{s}{e} \mid \unpair{\alpha}{x}{e}{e}
  %&\mid \mathrlap{\eq{e}{\tau}{e} \mid \refl{e} \mid \J{P}{d}{p}}
  \end{align*}
  \caption{Syntax (base \lang)}
  \label{#1}
  \end{figure}
}

\newcommand{\FigRed}[1]{
  \begin{figure}[h]
  \centering
  \begin{mathpar}
  \fbox{$\red{\Phi; \Gamma}{e}{e}$} \qquad
  \fbox{$\red*{\Phi; \Gamma}{e}{e}$} \hfill \\
  \inferrule[]{
    (\define{x}{\tau}{e}) \in \Gamma
  }{
    \red{\Phi; \Gamma}{x}{e}
  }
  \and \inferrule[]{~}{\red{\Phi; \Gamma}{\app{(\fun{x}{\tau}{e})}{e'}}{\subst{e}{x}{e'}}}
  %\and \inferrule[]{~}{\red{\Phi; \Gamma}{\J{}{d}{\refl{}}}{d}}
  \and \inferrule[]{~}{\red{\Phi; \Gamma}{\App{(\Fun{\alpha}{e})}{s}}{\subst{e}{\alpha}{s}}}
  \and \inferrule[]{~}{\red{\Phi; \Gamma}{\App{(\Fun<{\alpha}{r}{e})}{s}}{\subst{e}{\alpha}{s}}}
  \and \inferrule[]{~}{\red{\Phi; \Gamma}{\letin{x}{\tau}{e'}{e}}{\subst{e}{x}{e'}}}
  %\and \inferrule[]{~}{\red{\Phi; \Gamma}{\unpair{\alpha}{x}{\Pair{s}{e'}}{e}}{\subst{e}{\alpha, x}{s, e'}}}
  \\\\
  \inferrule[\rlabel{$\rhd^*$-once}{red*-once}]{
    \red{\Phi; \Gamma}{e_1}{e_2}
  }{
    \red*{\Phi; \Gamma}{e_1}{e_2}
  }
  \and
  \inferrule[\rlabel{$\rhd^*$-refl}{red*-refl}]{~}{\red*{\Phi; \Gamma}{e}{e}}
  \and
  \inferrule[\rlabel{$\rhd^*$-trans}{red*-trans}]{
    \red*{\Phi; \Gamma}{e_1}{e_2} \\
    \red*{\Phi; \Gamma}{e_2}{e_3}
  }{
    \red*{\Phi; \Gamma}{e_1}{e_3}
  }
  \and
  \inferrule[\rlabel{$\rhd^*$-cong}{red*-cong}]{
    \text{For every $1 \leq i \leq n$:} \and
    \red*{\Phi'; \Gamma'}{e_i}{e'_i}
  }{
    \red*{\Phi; \Gamma}{\subst{e}{x_1, \seq, x_n}{e_1, \seq, e_n}}{\subst{e}{x_1, \seq, x_n}{e'_1, \seq, e'_n}}
  }
  \end{mathpar}
  \caption{Reduction rules (base \lang)}
  \label{#1}
  \end{figure}
}

\newcommand{\FigSubtype}[1]{
  \begin{figure}[h]
  \centering
  \begin{mathpar}
  \fbox{$\subtype{\Phi; \Gamma}{\tau}{\tau}$} \qquad
  \fbox{$\acum{e}{e}$} \hfill \\
  \inferrule[\rlabel{$\preccurlyeq$-red}{subtype-red}]{
    \red*{\Phi; \Gamma}{\tau_1}{\sigma_1} \\
    \red*{\Phi; \Gamma}{\tau_2}{\sigma_2} \\
    \acum{\sigma_1}{\sigma_2}
  }{
    \subtype{\Phi; \Gamma}{\tau_1}{\tau_2}
  }
  \and
  \inferrule[\rlabel{$\sqsubseteq$-prop}{acum-prop}]{~}{\acum{\Prop{}}{\Type{i}}}
  \and
  \inferrule[\rlabel{$\sqsubseteq$-type}{acum-type}]{i \leq j}{\acum{\Type{i}}{\Type{j}}}
  \and
  \inferrule[\rlabel{$\sqsubseteq$-refl}{acum-refl}]{
  }{
      \acum{e}{e}
  }
  \and
  \inferrule[\rlabel{$\sqsubseteq$-pi}{acum-pi}]{
    \acum{\tau_1}{\tau_2}
  }{
    \acum{\funtype{x}{\sigma}{\tau_1}}{\funtype{x}{\sigma}{\tau_2}}
  }
  \and
  \inferrule[\rlabel{$\sqsubseteq$-forall}{acum-forall}]{
    \acum{\tau_1}{\tau_2}
  }{
    \acum{\Funtype{\alpha}{\tau_1}}{\Funtype{\alpha}{\tau_2}}
  }
  \and
  \inferrule[\rlabel{$\sqsubseteq$-forall$<$}{acum-forall<}]{
    \acum{\tau_1}{\tau_2}
  }{
    \acum{\Funtype<{\alpha}{s}{\tau_1}}{\Funtype<{\alpha}{s}{\tau_2}}
  }
  \end{mathpar}
  \caption{Subtyping and $\alpha$-cumulativity rules}
  \label{#1}
  \end{figure}
}

\newcommand{\FigSubsize}[1]{
  \begin{figure}[h]
  \centering
  \begin{mathpar}
  \fbox{$\wf{\Phi}{s}$} \qquad
  \fbox{$\subsize{\Phi}{s}{s}$} \hfill \\
  \inferrule[]{
    \wf{}{\Phi} \\
    \alpha \in \Phi
    \textit{ or }
    (\bound{\alpha}{s}) \in \Phi
  }{
    \wf{\Phi}{\alpha}
  }
  \and
  \inferrule[]{\wf{}{\Phi}}{\wf{\Phi}{\circ}}
  \and
  \inferrule[]{
    \wf{\Phi}{s}
  }{
    \wf{\Phi}{\sss{s}}
  }
  \and
  \inferrule[]{
    \wf{}{\Phi} \\
    (\bound{\alpha}{s}) \in \Phi
  }{
    \subsize{\Phi}{\sss{\alpha}}{s}
  }
  \\
  \inferrule[]{
    \wf{\Phi}{s}
  }{
    \subsize{\Phi}{\circ}{s}
  }
  \and
  \inferrule[]{
    \wf{\Phi}{s}
  }{
    \subsize{\Phi}{s}{s}
  }
  \and
  \inferrule[]{
    \wf{\Phi}{s}
  }{
    \subsize{\Phi}{s}{\sss{s}}
  }
  \and
  \inferrule[]{
    \subsize{\Phi}{r}{s}
  }{
    \subsize{\Phi}{\sss{r}}{\sss{s}}
  }
  \and
  \inferrule[]{
    \subsize{\Phi}{s_1}{s_2} \\\\
    \subsize{\Phi}{s_2}{s_3}
  }{
    \subsize{\Phi}{s_1}{s_3}
  }
  % \and \inferrule[]{~}{\subsize{\Gamma}{s}{\infty}}
  \end{mathpar}
  \caption{Size and subsizing rules}
  \label{#1}
  \end{figure}
}

\newcommand{\FigWF}[1]{
  \begin{figure}[h]
  \centering
  \begin{mathpar}
  \fbox{$\wf{}{\Phi}$} \qquad
  \fbox{$\wf{\Phi}{\Gamma}$} \hfill \\
  \inferrule[\rlabel{nil}{nil}]{~}{\wf{}{\mt}}
  \and
  \inferrule[\rlabel{cons-size}{cons-size}]{
    \wf{}{\Phi}
  }{
    \wf{}{\Phi, \alpha}
  }
  \and
  \inferrule[\rlabel{cons-size$<$}{cons-size<}]{
    \wf{}{\Phi} \\\\
    \wf{\Phi}{s}
  }{
    \wf{}{\Phi, \bound{\alpha}{s}}
  }
  \and
  \inferrule[nil]{\wf{}{\Phi}}{\wf{\Phi}{\mt}}
  \and
  \inferrule[\rlabel{cons-ass}{cons-ass}]{
    \wf{\Phi}{\Gamma} \\\\
    \infer{\Phi; \Gamma}{\tau}{U}
  }{
    \wf{\Phi}{\Gamma, \annot{x}{\tau}}
  }
  \and
  \inferrule[\rlabel{cons-def}{cons-def}]{
    \wf{\Phi}{\Gamma} \\\\
    \infer{\Phi; \Gamma}{e}{\tau}
  }{
    \wf{\Phi}{\Gamma, \define{x}{\tau}{e}}
  }
  \end{mathpar}
  \caption{Well-formedness rules}
  \label{#1}
  \end{figure}
}

\newcommand{\FigRulesAxioms}[1]{
  \begin{figure}[h]
  \centering
  \begin{align*}
  \axioms{\Prop} &= \Type{1} &
  \rules{U}{\Prop} &= \Prop \\
  \axioms{\Type{i}} &= \Type{i+1} &
  \rules{\Prop}{U} &= U \\
  && \rules{\Type{i}}{\Type{j}} &= \Type{\maximum{i, j}}
  \end{align*}
  \caption{Some metafunctions}
  \label{#1}
  \end{figure}
}

\newcommand{\FigTyping}[1]{
  \begin{figure}[p]
  \centering
  \begin{mathpar}
  \fbox{$\type{\Phi; \Gamma}{e}{\tau}$} \hfill \\
  \inferrule[\rlabel*{conv}]{
    \infer{\Phi; \Gamma}{e}{\sigma} \\
    \check{\Phi; \Gamma}{\sigma}{U} \\\\
    \infer{\Phi; \Gamma}{\tau}{U} \\
    \subtype{\Phi; \Gamma}{\sigma}{\tau}
  }{
    \check{\Phi; \Gamma}{e}{\tau}
  }
  \and
  \inferrule[\rlabel*{var}]{
    \wf{\Phi}{\Gamma} \\
    (\annot{x}{\tau}) \in \Gamma \\\\
    \textit{or } (\define{x}{\tau}{e}) \in \Gamma
  }{
    \infer{\Phi; \Gamma}{x}{\tau}
  }
  \and
  \inferrule[\rlabel*{univ}]{
    \wf{\Phi}{\Gamma} \\
  }{
    \infer{\Phi; \Gamma}{U}{\axioms{U}}
  }
  \and
  \inferrule[\rlabel*{pi}]{
    \infer{\Phi; \Gamma}{\sigma}{U_1} \\
    \infer{\Phi; \Gamma, \annot{x}{\sigma}}{\tau}{U_2}
  }{
    \infer{\Gamma}{\funtype{x}{\sigma}{\tau}}{\rules{U_1}{U_2}}
  }
  \and
  \inferrule[\rlabel*{lam}]{
    \infer{\Phi; \Gamma}{\sigma}{U} \\
    \infer{\Phi; \Gamma, \annot{x}{\sigma}}{e}{\tau}
  }{
    \infer{\Phi; \Gamma}{\fun{x}{\sigma}{e}}{\funtype{x}{\sigma}{\tau}}
  }
  \and
  \inferrule[\rlabel*{app}]{
    \infer{\Phi; \Gamma}{e_1}{\funtype{x}{\sigma}{\tau}} \\
    \check{\Phi; \Gamma}{e_2}{\sigma}
  }{
    \infer{\Phi; \Gamma}{\app{e_1}{e_2}}{\subst{\tau}{x}{e_2}}
  }
  \and
  \inferrule[\rlabel*{let}]{
    \infer{\Phi; \Gamma}{\sigma}{U} \\
    \check{\Phi; \Gamma}{e_1}{\sigma} \\\\
    \infer{\Phi; \Gamma, \define{x}{\sigma}{e_1}}{e_2}{\tau}
  }{
    \infer{\Phi; \Gamma}{\letin{x}{\sigma}{e_1}{e_2}}{\subst{\tau}{x}{e_1}}
  }
  \and
  \inferrule[\rlabel*{forall}]{
    \infer{\Phi, \alpha; \Gamma}{\tau}{U}
  }{
    \infer{\Phi; \Gamma}{\Funtype{\alpha}{\tau}}{U}
  }
  \and
  \inferrule[\rlabel*{slam}]{
    \infer{\Phi,\alpha; \Gamma}{e}{\tau}
  }{
    \infer{\Phi; \Gamma}{\Fun{\alpha}{e}}{\Funtype{\alpha}{\tau}}
  }
  \and
  \inferrule[\rlabel*{sapp}]{
    \infer{\Phi; \Gamma}{e}{\Funtype{\alpha}{\tau}} \\
    \wf{\Phi}{s}
  }{
    \infer{\Phi; \Gamma}{\App{e}{s}}{\subst{\tau}{\alpha}{s}}
  }
  \and
  \inferrule[\rlabel{forall$<$}{forall<}]{
    \wf{\Phi}{s} \\
    \infer{\Phi, \bound{\alpha}{s}; \Gamma}{\tau}{U}
  }{
    \infer{\Phi; \Gamma}{\Funtype<{\alpha}{s}{\tau}}{U}
  }
  \and
  \inferrule[\rlabel{slam$<$}{slam<}]{
    \wf{\Phi}{s} \\
    \infer{\Phi, \bound{\alpha}{s}; \Gamma}{e}{\tau}
  }{
    \infer{\Phi; \Gamma}{\Fun<{\alpha}{s}{e}}{\Funtype<{\alpha}{s}{\tau}}
  }
  \and
  \inferrule[\rlabel{sapp$<$}{sapp<}]{
    \infer{\Phi; \Gamma}{e}{\Funtype<{\alpha}{r}{\tau}} \\
    \subsize{\Phi}{\hat{s}}{r}
  }{
    \infer{\Phi; \Gamma}{\App{e}{s}}{\subst{\tau}{\alpha}{s}}
  }
  \iffalse
  \and
  \inferrule[\rlabel*{exists}]{
    \infer{\Phi, \alpha; \Gamma}{\tau}{U}
  }{
    \infer{\Phi; \Gamma}{\Pairtype{\alpha}{\tau}}{U}
  }
  \and
  \inferrule[\rlabel*{pair}]{
    \wf{\Phi}{s} \\
    \check{\Phi; \Gamma}{e}{\subst{\tau}{\alpha}{s}}
  }{
    \infer{\Phi; \Gamma}{\Pair{s}{e}_{\Pairtype{\alpha}{\tau}}}{\Pairtype{\alpha}{\tau}}
  }
  \and
  \inferrule[\rlabel*{unpair}]{
    \infer{\Phi; \Gamma}{e_1}{\Pairtype{\alpha}{\sigma}} \\
    \infer{\Phi, \alpha; \Gamma, \annot{x}{\sigma}}{e_2}{\tau}
  }{
    \infer{\Phi; \Gamma}{\unpair{\alpha}{x}{e_1}{e_2}}{\tau}
  }
  \and
  \inferrule[\rlabel*{eq}]{
    \infer{\Phi; \Gamma}{\tau}{U} \\
    \check{\Phi; \Gamma}{e_1}{\tau} \\
    \check{\Phi; \Gamma}{e_2}{\tau}
  }{
    \infer{\Phi; \Gamma}{\eq{e_1}{\tau}{e_2}}{U}
  }
  \and
  \inferrule[\rlabel*{refl}]{
    \infer{\Phi; \Gamma}{e}{\tau}
  }{
    \infer{\Phi; \Gamma}{\refl{e}}{\eq{e}{\tau}{e}}
  }
  \and
  \inferrule[\rlabel*{J}]{
    \infer{\Phi; \Gamma}{p}{\eq{e_1}{\tau}{e_2}} \\
    \fresh{y, z} \\\\
    \infer{\Phi; \Gamma, \annot{y}{\tau}, \annot{z}{\eq{e_1}{\tau}{y}}}{\app{\app{P}{y}}{z}}{U} \\
    \check{\Phi; \Gamma}{d}{\app{\app{P}{e_1}}{\refl{e_1}}} \\
  }{
    \infer{\Phi; \Gamma}{\J{P}{d}{p}}{\app{\app{P}{e_2}}{p}}
  }
  \fi
  \end{mathpar}
  \caption{Typing rules}
  \label{#1}
  \end{figure}
}

% TODO: remember to mention that new notation not in a figure is highlighted in grey

\section{Base \lang}

Although sized types are quite pointless without any inductive types to be sized,
I present in this subsection the sublanguage of \lang without naturals or wellfounded trees
to not only get the preliminary details out of the way first,
but also to show that the sublanguage is independent of the chosen inductive types.

\FigSyntax{fig:syntax}
\cref{fig:syntax} gives its syntax, which consists of universes $U$,
sizes $s$, and terms $e$, which includes term functions, size abstractions,
and the propositional equality type.
Most judgements use two environments: a term environment $\Gamma$ with assumptions $\annot{x}{\tau}$
and definitions $\define{x}{e}$, and a size environment $\Phi$ with unbounded and bounded size variables.
I also use the assumption environment%
\footnote{These are conventionally called \emph{telescopes}\index{telescopes} due to \citet{telescope}.}
$\Delta$ as a shorthand when writing nested expressions with assumptions;
in particular, letting for instance $\Delta_{xy} = \annot{x}{\sigma_1}, \annot{y}{\sigma_2}$,
I use \new{$\arr*{\Delta_{xy}}{\tau}$} to mean $\funtype{x}{\sigma_1}{\funtype{y}{\sigma_2}{\tau}}$.
As a loose convention, I use $\tau, \sigma$ for type-like terms,
$P$ for the \emph{motive}\index{motive} of eliminators,
$p$ for proofs of equality, and
$f, g$ for variables representing functions.

\FigRed{fig:reduction}
The reduction rules and their reflexive, transitive, congruent closures are described in \cref{fig:reduction}.
\new{$\subst{e}{x}{e'}$} denotes capture-avoiding substitution of $x$ for $e'$ within $e$,
and \new{$\subst{e}{x_1, \seq, x_n}{e_1, \seq, e_n}$} correspondingly denotes simultaneous substitution.
For every syntactic form of a term there is a corresponding congruence rule,
which is summarized by \rref{red*-cong};
the full set of rules can be found in \TODO. % put them in the appendix
By convention, the reduction rules for functions are also referred to as $\beta$-reduction,
for $\kw{let}$ expressions as $\zeta$-reduction,
and for defined variables as $\delta$-reduction.%
\footnote{In MLTT, all rules consisting of an elimination form around an introduction form
are referred to as $\beta$-reduction,
which would include the reduction rule for $\J*$, not just functions.
On the other hand, in type theories where propositional equality is defined as a inductive type,
the reduction rule would be called $\iota$-reduction.
I avoid this dilemma by not referring to the reduction rule for $\J*$ by name at all.}

\FigSubtype{fig:subtyping}


\clearpage
\FigSubsize{fig:subsizing}
\FigWF{fig:wf}