\documentclass[acmsmall,nonacm]{acmart}
\acmConference[]{}{}{}

\usepackage{minted}
\newmintinline[agda]{agda}{escapeinside=<>,mathescape=true}

\newcommand{\α}{\alpha}
\newcommand{\β}{\beta}
\newcommand{\γ}{\gamma}
\newcommand{\σ}{\sigma}
\newcommand{\Γ}{\Gamma}
\newcommand{\Λ}{\Lambda}
\newcommand{\∀}{\forall}
\newcommand{\↑}{\uparrow}
\newcommand{\⊢}{\vdash}
\newcommand{\↦}{\mapsto}
\newcommand{\⊳}{\rhd}
\newcommand{\fix}{\textrm{fix}}

% Each submission (referred to as “abstract” below) should include the student author’s name and e-mail address;
% institutional affiliation; research advisor’s name; ACM student member number; category (undergraduate or graduate)
\author{Jonathan Chan}
\title{A Syntactic Model of Sized Dependent Types}

\begin{document}
\maketitle

\section{???}

The types-as-propositions paradigm associates certain type theories with formal logical systems,
and consequently types in those theories with propositions in those logics.
Furthermore, well-typed programs are associated with proofs of the corresponding proposition.
Many dependent type theories, for instance, correspond to higher-order logics,
and having an automated type checker means having the ability to automatically verify proofs.
One must be careful, however, not to allow nonterminating programs,
because they correspond to logical inconsistencies, i.e. proofs of falsehood;
additionally, in dependent type checkers where programs may be evaluated during type checking,
failure to rule out nonterminating programs leads to nonterminating type checking.
Contemporary proof assistants based on dependent type theories, such as Coq, Agda, Lean, Idris,
and more, typically restrict recursive functions to \emph{structurally-recursive} ones,
where the argument of recursive calls must be \emph{syntactically} smaller.
As an example, consider the following function \agda{minus n m} that computes $\min(n - m, 0)$.

\begin{minted}{agda}
minus : Nat → Nat → Nat
minus Zero m = Zero
minus n Zero = n
minus (Succ n) (Succ m) = minus n m
\end{minted}

Inuitively, any well-typed call to \agda{minus} is guaranteed to terminate because the recursive call
only occurs on constructor arguments, and the base cases must be reached eventually.
However, we often wish to write terminating functions that aren't necessarily syntactically recursive.
Consider now the following function \agda{div n m} that computes $\big\lceil\frac{n}{m+1}\big\rceil$ using \agda{minus}.

\begin{minted}{agda}
div : Nat → Nat → Nat
div Zero m = Zero
div (Succ n) m = Succ (div (minus n m) m)
\end{minted}

Despite the fact that \agda{minus} returns a natural no greater than its first argument,
meaning that the first argument of the recursive call to \agda{div} has a smaller magnitude
than the original argument,
the type checker is unable to conclude that this is a terminating function,
because the first argument isn't \emph{syntactically} smaller (i.e. \agda{n}).
The programmer would have to alter the definition of \agda{minus} to prove this property,
as well as the definition of \agda{div} to use this proof, making writing otherwise simple code burdensome.
Sometimes it would be possible to make the type checker a little more clever by selectively inlining
certain functions that allow the syntactic termination check to pass,
but this is not always possible, and requiring inlining just to pass termination checking is anti-modular.
If inlining \agda{minus} in the body of \agda{div} helped, then we are forced to import its entire definition,
rather than merely its name and type, as is only required for type checking alone.

%\paragraph*{}
A \emph{type-based} rather than syntactic method of termination checking uses \emph{sized types} (Hughes 1996),
where a function is guaranteed to terminate simply if it type checks.
To use sized types, we first alter our definition of naturals to take a size expression as a parameter.

\begin{minted}[escapeinside=<>,mathescape=true]{agda}
data Nat [<$\α$>] : Type where
  Zero : <$\∀ \β < \α$>. Nat [<$\α$>]
  Succ : <$\∀ \β < \α$>. Nat [<$\β$>] → Nat [<$\β$>]
\end{minted}

The bounded size quantification \agda{<$\∀\β < \α$>} asks for some size $\β$ strictly smaller than $\α$.
So to construct the successor of some natural \agda{n : Nat [s]}, we need a size larger than \agda{s}.
Let \agda{<$\↑$>s} denote the next size up from \agda{s}.
Then the successor is simply \agda{Succ {<$\↑$>s} [s] n} (using curly braces for the implicit size parameter).
Now, let us express the fact that \agda{minus} returns a natural no larger than its first argument using sizes,
i.e. that \agda{minus} is in fact \emph{size-preserving}.

\begin{minted}[escapeinside=<>,mathescape=true]{agda}
minus : <$\∀\α$>, <$\β$>. Nat [<$\α$>] → Nat [<$\β$>] → Nat [<$\α$>]
\end{minted}

Then we are able to write a \agda{div} function that the type checker will accept to be terminating
with just a few more annotations.

\begin{minted}[escapeinside=<>,mathescape=true]{agda}
div : <$\∀\α$>, <$\β$>. Nat [<$\α$>] → Nat [<$\β$>] → Nat [<$\α$>]
div [<$\α$>] [<$\β$>] (Zero {<$\α$>} [<$\γ$>])   m = Zero {<$\α$>} [<$\γ$>]
div [<$\α$>] [<$\β$>] (Succ {<$\α$>} [<$\γ$>] n) m = Succ {<$\α$>} [<$\γ$>] (div [<$\γ$>] [<$\β$>] (minus [<$\γ$>] [<$\β$>] n m) m)
\end{minted}

In the second branch, when matching on the first argument of size $\α$, we have the successor of \agda{n},
which itself has size $\γ < \α$.
Then the call to \agda{minus} returns a natural of size $\γ$, which is then passed to the recursive call of \agda{div}.
Type checking passes without problem because the size $\γ$ of the recursive call is strictly smaller
than the size $\α$ of the original call to \agda{div}.
More precisely, when expressing recursive functions as fixpoints,
their typing rule and reduction behaviour is the following
(using $e[x \↦ e']$ to denote substitution of $x$ by $e'$ in $e$):
\begin{align*}
&\Γ, \α, f : \∀\β < \α \mathpunct{.} \σ[\α \↦ \β] \⊢ e : \σ \\[-2\jot]
&\cline{1-2}
&\Γ \⊢ \fix \, f \, [\α] : \σ \coloneqq e : \∀\α \mathpunct{.} \σ
\end{align*}
\begin{align*}
(\fix \, f \, [\α] : \σ \coloneqq e) \: [s] \: e' &\⊳ e[\α \↦ s, f \↦ \Λ\β < s \mathpunct{.} (\fix \, f \, [\α] : \σ \coloneqq e) [\β]] \: e'
\end{align*}

A fixpoint is well-typed if the body has the same type under the environment where
the recursive reference to the fixpoint quantifies over a \emph{smaller} size.
When reducing the fixpoint applied to some size, we substitute in the body the fixpoint bounded by that size.

\section{???}

Once we have a program and correctness proofs about the program, we may wish to compile and run our program.
In Coq and Agda, programs get \emph{extracted} to OCaml and Haskell code, respectively, which are then compiled.
However, the process of extraction necessarily discards some type information,
since the extraction targets are not dependently typed.
This makes it possible to link our proven-safe program after extraction with unsafe code,
which might cause runtime errors, making all of the work proving it correct in vain.
To ensure that our proofs are preserved and verified even during linking,
we want to compile our code in a \emph{type-preserving} manner (Bowman, 2018).
The type-based nature of sized types makes it more amenable than syntactic termination checking
to type-preserving compilation of recursive functions.

Traditionally in dependent type theories, there are two mutually-dependent judgements:
the typing judgement, and the (typed) equality judgement.
Equality is used in the following typing rule:
\end{document}