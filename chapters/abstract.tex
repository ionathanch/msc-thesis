\chapter{Abstract}

Many contemporary proof assistants based on dependent type theories such as Coq and Agda
are founded on the types-as-propositions paradigm where type checking a program
corresponds to verifying a proof of some proposition in a higher-order predicate logic.
To ensure decidability of type checking and consistency of the logic,
these proof assistants forbid nonterminating recursive functions
using guard predicates that only allow structurally recursive functions
recurring only on syntactically smaller arguments.
However, these guard predicates are sometimes too restrictive
and reject obviously-terminating functions that aren't otherwise structurally recursive,
creating extra work for the programmer to convince the guard checker.

An alternative is to use type-based termination checking such as sized types,
where inductively-defined types are annotated with sizes.
Successful type checking guarantees that functions recur only on arguments whose types have smaller sizes,
rather than merely on syntactic subarguments.
There exist many sized dependent type theories,
but none simultaneously feature higher-rank size quantification,
which allows for passing around size-preserving functions,
and bounded size quantification,
which eliminates the need for complex monotonicity checks required by prior sized type systems.

In this thesis, I design a sized dependent type theory with higher-rank and bounded sizes (\lang),
and show that it's suitable for theorem proving by proving its logical consistency with a syntactic model:
by compiling \lang into the Extensional Calculus of Inductive Constructions (\CICE),
a variant of Coq's core type theory,
and showing that this translation is type preserving,
the consistency of \lang follows from the consistency of \CICE.
This approach, unfortunately, refutes the existence of an ``infinite'' size strictly greater than all sizes,
which is present in prior sized type systems to overcome the limitations of finitary size expressions,
meaning that some infinitary constructs aren't definable in \lang.

Even so, \lang provies a valid foundation for sized types in a proof assistant,
opening the way for future work on recovering expressivity lost from the lack of an infinite size
and on restricting sized types in Agda,
the only (inconsistent!) implementation of sized types in a major proof assistant,
to be consistent.