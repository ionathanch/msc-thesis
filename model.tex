\chapter{Syntactic Model of \lang}

\newcommand{\FigSyntaxCIC}[1]{
  \begin{figure}[h]
  \centering
  \begin{align*}
  \iT, \jT, \kT, \mT, \nT &\Coloneqq \meta{\textsf{naturals}} &
  \GammaT &\Coloneqq \mt \mid \GammaT, \annotT{\xT}{\tauT} \mid \GammaT, \define{\xT}{\eT} \\
  \fT, \gT, \xT, \yT, \zT &\Coloneqq \meta{\textsf{term variables}} &
  \DeltaT &\Coloneqq \mt \mid \DeltaT, \annotT{\xT}{\tauT} \\
  \xT &\Coloneqq \meta{\textsf{inductive type names}} &
  \UT &\Coloneqq \PropT \mid \TypeT{\iT} \\
  \cT &\Coloneqq \meta{\textsf{inductive constructor names}} &
  \dT &\Coloneqq \app{\dataT{\app{\XT}{\DeltaT}}{\arrT*{\DeltaT}{\UT}}}{\DeltaT} \\
  \eT, \dT, \pT, \pT, \tauT, \sigmaT &\Coloneqq
    \mathrlap{\xT \mid \UT \mid \funtypeT{\xT}{\tauT}{\tauT} \mid \funT{\xT}{\tauT}{\eT} \mid \app{\eT}{\eT} \mid \letinT{\xT}{\tauT}{\eT}{\eT} \mid \fixT{\iT}{\fT}{\tauT}{\eT}} \\
  &\mid \mathrlap{\eqT{\eT}{\tauT}{\eT} \mid \reflT{\eT} \mid \JT{\PT}{\dT}{\pT} \mid \matchT{\eT}{\funT*{(\yT \seq).\xT}{\PT}}{(\app{\cT}{\zT}{\seq} \RightarrowT \eT) \seq}}
  \end{align*}
  \caption{Syntax (\CICE)}
  \label{#1}
  \end{figure}
}

\newcommand{\FigEquiv}[1]{
  \begin{figure}
  \centering
  \begin{mathpar}
  \fbox{$\defeq{\GammaT}{\eT}{\eT}{\tauT}$} \hfill \\
  \inferrule[\rlabel{$\equiv$-refl}{equiv-refl}]{
    \type{\GammaT}{\eT}{\tauT}
  }{
    \defeq{\GammaT}{\eT}{\eT}{\tauT}
  }
  \and
  \inferrule[\rlabel{$\equiv$-sym}{equiv-sym}]{
    \defeq{\GammaT}{\eT_2}{\eT_1}{\tauT}
  }{
    \defeq{\GammaT}{\eT_1}{\eT_2}{\tauT}
  }
  \and
  \inferrule[\rlabel{$\equiv$-trans}{equiv-trans}]{
    \defeq{\GammaT}{\eT_1}{\eT_2}{\tauT} \\\\
    \defeq{\GammaT}{\eT_2}{\eT_3}{\tauT}
  }{
    \defeq{\GammaT}{\eT_1}{\eT_3}{\tauT}
  }
  \and
  \inferrule[\rlabel{$\equiv$-conv}{equiv-conv}]{
    \subtype{\GammaT}{\sigmaT}{\tauT} \\\\
    \defeq{\GammaT}{\eT_1}{\eT_2}{\sigmaT}
  }{
    \defeq{\GammaT}{\eT_1}{\eT_2}{\tauT}
  }
  \and
  \inferrule[\rlabel{$\equiv$-reflect}{equiv-reflect}]{
    \type{\GammaT}{\pT}{\eqT{\eT_1}{\tauT}{\eT_2}}
  }{
    \defeq{\GammaT}{\eT_1}{\eT_2}{\tauT}
  }
  \and
  \inferrule[\rlabel{$\equiv$-cong}{equiv-conv}]{
    \textrm{For every $1 \leq i \leq n$:} \\
    \defeq{\GammaT'}{\eT_i}{\eT'_i}{\tauT'}
  }{
    \defeq{\GammaT}
      {\subst{\eT}{\xT_1, \seq, \xT_n}{\eT_1, \seq, \eT_n}}
      {\subst{\eT}{\xT_1, \seq, \xT_n}{\eT'_1, \seq, \eT'_n}}
      {\tauT}
  }
  \and
  \inferrule[\rlabel{$\equiv$-$\delta$}{equiv-delta}]{
    (\defineT{\xT}{\eT}) \in \GammaT \\
    (\annotT{\xT}{\tauT}) \in \GammaT
  }{
    \defeq{\GammaT}{\xT}{\eT}{\tauT}
  }
  \and
  \inferrule[\rlabel{$\equiv$-$\beta$}{equiv-beta}]{
    \type{\GammaT}{\sigmaT}{\UT} \\
    \type{\GammaT, \annotT{\xT}{\sigmaT}}{\eT}{\tauT} \\
    \type{\GammaT}{\eT'}{\sigmaT}
  }{
    \defeq{\GammaT}{\app{(\funT{\xT}{\sigmaT}{\eT})}{\eT'}}{\subst{\eT}{\xT}{\eT'}}{\tauT}
  }
  \and
  \inferrule[\rlabel{$\equiv$-$\eta$}{equiv-eta}]{
    \defeq{\GammaT, \annotT{\xT}{\sigmaT}}{\app{\eT_1}{\xT}}{\app{\eT_2}{\xT}}{\tauT}
  }{
    \defeq{\GammaT}{\eT_1}{\eT_2}{\funtypeT{\xT}{\sigmaT}{\tauT}}
  }
  \and
  \inferrule[\rlabel{$\equiv$-$\zeta$}{equiv-zeta}]{
    \type{\GammaT}{\eT'}{\sigmaT} \\
    \type{\GammaT, \annotT{\xT}{\sigmaT}, \defineT{\xT}{\eT'}}{\eT}{\tauT}
  }{
    \defeq{\GammaT}{\letinT{\xT}{\sigmaT}{\eT'}{\eT}}{\subst{\eT}{\xT}{\eT'}}{\subst{\tauT}{\xT}{\eT'}}
  }
  \and
  \inferrule[\rlabel{$\equiv$-$\rho$}{equiv-rho}]{
    \type{\GammaT}{\eT}{\tauT} \\
    \fresh{\yT, \zT} \\
    \type{\GammaT, \annotT{\yT}{\tauT}, \annotT{\zT}{\eqT{\eT}{\tauT}{\yT}}}{\app{\PT}{\yT}{\zT}}{\UT} \\
    \type{\GammaT}{\dT}{\app{\PT}{\eT}{\reflT{\eT}}}
  }{
    \defeq{\GammaT}{\JT{\PT}{\dT}{\reflT{\eT}}}{\dT}{\app{\PT}{\eT}{\reflT{\eT}}}
  }
  \and
  \inferrule[\rlabel{$\equiv$-$\iota$}{equiv-iota}]{\cdots}{
    \defeq{\GammaT}{\matchT{\app{\cT}{\eT_1}{\seq}{\eT_m}}{\any}{\seq(\app{\cT}{\zT_1}{\seq}{\zT_m} \RightarrowT \eT)\seq}}{\subst{\eT}{\zT_1, \seq, \zT_m}{\eT_1, \seq, \eT_m}}{\tauT}
  }
  \and
  \inferrule[\rlabel{$\equiv$-$\mu$}{equiv-mu}]{\cdots}{
    \defeq{\GammaT}{\app{(\fixT{\iT}{\fT}{\tauT}{\eT})}{\eT'_1}{\seq}{\eT'_n}{(\app{\cT}{\eT_1}{\seq}{\eT_m})}}{\app{\subst{\eT}{\fT}{\fixT{\iT}{\fT}{\tauT}{\eT}}}{\eT'_1}{\seq}{\eT'_n}{(\app{\cT}{\eT_1}{\seq}{\eT_m})}}{\tauT}
  }
  \end{mathpar}
  \caption{Equivalence rules}
  \label{#1}
  \end{figure}
}

In addition to the notation used in \cref{ch:sized-dep-types},
given variables $\vec{\xT} = \xT_1 \seq \xT_n$,
terms $\vec{\eT} = \eT_1 \seq \eT_n$,
and types $\vec{\tauT} = \tauT_1 \seq \tauT_n$,
\new{$\annotT{\vec{\xT}}{\vec{\tauT}}$} denotes the assumption environment
$\annotT*{\xT_1}{\tauT_1}, \seq, \annotT*{\xT_n}{\tauT_n}$, and
\new{$\subst{\eT}{\vec{\xT}}{\vec{\eT}}$} denotes the simultaneous substitution
$\subst{\eT}{\xT_1, \seq, \xT_n}{\eT_1, \seq, \eT_n}$.

\section{Target Type Theory}

\FigSyntaxCIC{fig:syntax-cic}
The syntax of \CICE is given in \cref{fig:syntax-cic};
differences from \lang include a 1-based index for the recursive argument of fixpoint expressions,
$\tg{case}$ expression motives abstracted over the target's inductive type indices,
and a \emph{propositional equality}\index{propositional equality} type with the reflexivity constructor and $\JT*$ eliminator.
New inductive types are defined using data definitions $\DT$,
whose syntax resembles the informal presentation used in \cref{ch:sized-dep-types}.
Metavariable usage convention is roughly the same as for \lang,
with the addition of $\pT$ for inductive type parameters or proofs of equality
and $\aT$ for inductive type indices.

The well-formedness conditions on inductive data definitions,
such as well-typedness and \emph{strict positivity}\index{strict positivity},
are entirely standard, so I don't detail them here;
see pCuIC~\citep{pCuIC} for instance for a full description.
Inductive definitions in their full generality are not needed,
and nonmutual, nonnested inductives suffice.
Indeed, only five inductive definitions are used for the translation,
for representing sizes, their order, their well-foundedness,
and naturals and well-founded trees.

\FigEquiv{fig:equivalence}
\FigTypingCIC{fig:typing-cic}